%!TEX root=/home/ska124/Dropbox/Thesis/thes-full.tex
%% Copyright 1998 Pepe Kubon
%%
%% `two.tex' --- 2nd chapter for thes-full.tex, thes-short-tex from
%%               the `csthesis' bundle
%%
%% You are allowed to distribute this file together with all files
%% mentioned in READ.ME.
%%
%% You are not allowed to modify its contents.
%%

%%%%%%%%%%%%%%%%%%%%%%%%%%%%%%%%%%%%%%%%%%%%%%%%%
%
%     Chapter 2   
%
%%%%%%%%%%%%%%%%%%%%%%%%%%%%%%%%%%%%%%%%%%%%%%%%

\chapter{Changes from \textsf{sfuthesis.sty}}
\label{two}

\section{Page layout}

All the changes\index{changes@changes from \textsf{sfuthesis.sty}} in
this section are not required by the regulations%
\index{thesis regulations}; they were made for better visual impact.

The four margins\index{margins} were set by \textsf{sfuthesis.sty}%
\index{sfuthesis.sty@\textsf{sfuthesis.sty}} to the values recommended
in the regulations\index{thesis regulations}; these were kept
unchanged. The space between the header\index{header} and the text was
slightly reduced and made dependent on the chosen point size of the
font (10, 11, or 12)\footnote{This text is set with 11pt size, which
  seems to be the best choice overall from the three
  possibilities.}---it now corresponds to one empty line of text.

The space between the text and the footer\index{footer} is set to the result of
adding half an empty line to the value of the space between the header
and the text. This centers the text vertically on the page with
respect to the other material. If, for any reasons, you decide to
adjust the vertical placement of the text, only the value of
\verb+\headsep+\index{headsep@\texttt{headsep}} needs to be modified.

The first footnote\index{footnotes} is pushed slightly lower under the
horizontal bar which separates footnotes from the text (see below). In
addition,
small vertical space separates individual footnotes on the page.\label{foot}%
\footnote{This is an example of a second footnote on the page. I will
  make it longer than the first one. As can be seen, the two footnotes
  are separated by small vertical space.

  This is a new paragraph in the second footnote. There is no
  additional vertical space between
  paragraphs in a footnote.}

The vertical space before the
\verb+paragraph+\index{paragraph@\texttt{paragraph}} and
\verb+subparagraph+\index{subparagraph@\texttt{subparagraph}} sections
was slightly reduced from the original values set by \LaTeX{}. The new
values work better for the line spacing\index{line spacing} used in the text (see
Appendix~\ref{app:spacing}).

\section{Line spacing}
\label{spacing}

The style file defines three line spacing lengths:
\verb+\textstretch+, \verb+\tighttextstretch+, and
\verb+\doublestretch+. The first is used for the regular text. The
second, which basically corresponds to the single spacing%
\index{single spacing}on a typewriter, is used for footnotes (see
page~\pageref{foot}) and inside tables\index{spacing in tables} (see
Table~\ref{tab1}) and figures\index{spacing in figures} (see
Figure~\ref{fig1})---including captions\index{spacing in captions}.
The last corresponds to the double spacing\index{double spacing} on a
typewriter and is used for the thesis title\index{title}, as required by the
regulations\index{thesis regulations}.

\begin{figure}[htbp]
  \begin{center}
    This shows that single line spacing\\
    is used inside figures.
    \caption{Example of a figure\label{fig1}}
  \end{center}
\end{figure}
%
\vspace*{-.3in}
\begin{table}[htbp]
  \begin{center}
    \begin{tabular}[t]{|ll|}
      \hline
      This shows & the same thing\\
      for tables. & \\
      \hline
    \end{tabular}    
    \caption{Example of a table\label{tab1}}
  \end{center}
\end{table}

In addition, I recommend using single spacing also for
bibliography\index{spacing in bibliography} and index%
\index{spacing in index} (if included), as it is done in this
document.%
\footnote{The regulations give you some leeway in this matter; you can
  use either the regular spacing, or tighter spacing---the choice is
  up to you.}\index{thesis regulations} The corresponding files
\texttt{files/bibl.tex}\index{bibl.tex@\texttt{bibl.tex}} and
\texttt{files/ind.tex}\index{ind.tex@\texttt{ind.tex}} show how to specify
that single spacing be used.


\section{Front matter}\index{front matter}

\subsection{Title page}\index{Title page}

The title\index{title} itself is set in bold face, capitalized, and double
spaced. The vertical white space was changed from fixed to stretchable
(and partially shrinkable) to allow for a better fit of the
material---titles have varying amouns of lines and people can have
several previous degrees.

\subsection{Approval page}\index{Approval page} 

Essentially the same, only made easier to modify if the text doesn't
fit on a single page or if you want to change the amount of white
space between signatures.

\subsection{Dedication and Quotation pages}
\index{Dedication page}\index{Quotation page}

The \textsf{csthesis.sty}\index{csthesis.sty@\textsf{csthesis.sty}}
package has a built-in support for optional Dedication and Quotation
pages. Both dedication\index{dedication} and
quotation\index{quotation} are typeset flushed right, dedication in
\textit{italics} and quotation \textsl{slanted} (both are illustrated
in the full version of this document). The pages are not titled, but
they are numbered, and a corresponding entry is added to the Contents
table.

There are no constraints on the look of these pages imposed by the
regulations\index{thesis regulations}, you can change the design as
you see fit.

\subsection{Contents}\index{Contents}
\label{cont}

The table of Contents was reverted back to the original design (for
\textsf{report.sty}) from \LaTeX. There are no specific requirements
in the regulations\index{thesis regulations} as to how the table
should look like, but all the changes made to it in
\textsf{sfuthesis.sty}\index{sfuthesis.sty@\textsf{sfuthesis.sty}}---reducing
white space between chapters, putting dotted lines on the chapter
level, or redefining the behavior of the \verb+\appendix+%
\index{appendix@\texttt{\symbol{'134}appendix}}
command---looked decidedly awful and so I got rid of them.


\subsection{Preface}\index{Preface (Foreword)} 

The regulations\index{thesis regulations} allow for a Preface or
Foreword section, which could contain a short blurb about ``why the
author came to study the subject of the thesis.'' It's probably
useless because nobody will want to write anything like that, but it's
incorporated in the package for completeness. The optional inclusion
of Preface necessitated the modification of the \verb+\afterpreface+%
\index{appendix@\texttt{\symbol{'134}appendix}}
command of
\textsf{sfuthesis.sty}\index{sfuthesis.sty@\textsf{sfuthesis.sty}},
which was replaced by two commands: \verb+\lists+%
\index{lists@\texttt{\symbol{'134}lists}}, and
\verb+\beforetext+\index{beforetext@\texttt{\symbol{'134}beforetext}}.

\section{Back matter}\index{back matter}

\subsection{Appendices}\index{appendices}

As mentioned in Section~\ref{cont},
\textsf{sfuthesis.sty}\index{sfuthesis.sty@\textsf{sfuthesis.sty}}
redefines the \verb+\appendix+%
\index{appendix@\texttt{\symbol{'134}appendix}} command of \LaTeX.
Again, there does not seem to be any reason to do that (the
regulations are silent on the formatting of appendices); and the
original design looks better.  Thus, the present style reverts to
\LaTeX's original design, where appendices are treated as chapters and
numbered alphabetically. (The change in numbering is accomplished by
issuing the
\verb+appendix+\index{appendix@\texttt{\symbol{'134}appendix}} command
before the text of the first appendix.)

\subsection{Bibliography}\index{Bibliography}

The \textsf{csthesis.sty}\index{csthesis.sty@\textsf{csthesis.sty}}
package does not influence the typesetting of Bibliography directly,
but---as I said in Section~\ref{spacing}---I recommend changing the
line spacing\index{spacing in bibliography} from normal to single, as
was done in \texttt{files/bibl.tex}\index{bibl.tex@\texttt{bibl.tex}}.
See the (fake) Bibliography of this document as an example of the
result.

\subsection{Index}\index{Index}

The same comment applies to the optional Index section (see
\texttt{files/ind.tex}\index{ind.tex@\texttt{ind.tex}} for how to
adjust the line spacing\index{spacing in index} and the Index section
of the full version of this document for the result).



