%!TEX root=/home/ska124/Dropbox/Thesis/thes-full.tex
%% Copyright 1998 Pepe Kubon
%%
%% `05-introduction.tex' --- 1st chapter for thes-full.tex, thes-short-tex from
%%                the `csthesis' bundle

%%%%%%%%%%%%%%%%%%%%%%%%%%%%%%%%%%%%%%%%%%%%%%%%%
%
%       Chapter 1 
%
%%%%%%%%%%%%%%%%%%%%%%%%%%%%%%%%%%%%%%%%%%%%%%%%

\chapter{Introduction}
\label{sec:introduction}


\section{Dissertation Outline}
\label{sec:outline}


\section{Contributions}
\label{sec:summary}
We conduct the first in-depth study of triangulation, the first using four real low-resource languages with realistic data settings. As part of the dissertation, we also build the first translation systems for three of the four languages. Our best Haitian-Creole system outperform the best system from the Sixth Workshop on Machine Translation, 2011. 


\section{Setup}
\label{sec:setup}


Moses~\cite{Koehn:07} was used for all the experiments. To build our baseline systems, we followed the standard set of steps: generated bi-directional alignments using GIZA++ ~\cite{OchNey:03}, followed by phrase extraction using the --grow-diag-final-and heuristic. The decoder parameters were optimized using Minimum Error Rate Training ~\cite{Och:03} by minimizing BLEU loss on a development set. All scores reported are case-insensitive BLEU~\cite{Papineni:02}. All language models were generated using SRILM~\cite{Stolcke:02}.~\cite{Ken:11} was used for language model scoring when decoding.