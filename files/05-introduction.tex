%!TEX root=/home/ska124/Dropbox/Thesis/thes-full.tex
%% Copyright 1998 Pepe Kubon
%%
%% `05-introduction.tex' --- 1st chapter for thes-full.tex, thes-short-tex from
%%                the `csthesis' bundle

%%%%%%%%%%%%%%%%%%%%%%%%%%%%%%%%%%%%%%%%%%%%%%%%%
%
%       Chapter 1 
%
%%%%%%%%%%%%%%%%%%%%%%%%%%%%%%%%%%%%%%%%%%%%%%%%

\chapter{Introduction}
\label{sec:introduction}

\section{Why study Low-Resource?}
\label{sec:low_resource}
Statistical Machine Translation(SMT) has enabled several languages like French, Spanish, Finnish and others to have publicly available translation systems which can translate from and into these languages. Google Translate has 81 languages you can translate from and to. In 2003, we had none ! 


Having said that, more than 90\% of the World languages do not have a publicly available SMT system. Moreover, most of them have not been studied in the literature before. In Table~\ref{table:numberspeakers}, we observe that the major languages have way more number of speakers than the languages we study in this disseration. 
\begin{table*}
	\begin{tabular}{lr}
	\toprule
	Language & \#speakers \\
	\toprule
	French & 120M \\
	Spanish & 466M \\
	Mandarin Chinese & 1026M \\
	\midrule
	Haitian Kreyol & 12M \\
	Malagasy & 18M \\
	Mawu & 2M \\
	Manin & 2M \\
	\bottomrule
	\end{tabular}
	\caption{Number of speakers for Major and low-resource languages}
	\label{table:numberspeakers}
	%\small
	%\centering
\end{table*}

All the four languages also throw up linguistic challenges that are not seen in the major languages. For instance, Malagasy has influence from French and Arabic. While there are some loan words from French, the numbers are written right-to-left like Arabic. It also shows vocabulary overlap with Bantu. Diacritics are used but only in certain circumstances. On the other hand, Mawukakan and Maninkakan show a more frequent usage of diacritics. But the accent can be different depending on the placement of the word. Haitian Kreyol is a French-based Creole but does not share any vocabulary with Parasian French. When studying low-resource languages, every part of the standard SMT pipeline needs a rethink. How do we tokenize the text ? 


SMT uses data-driven models to translate sentences in a source language to a given target language. A phrase-based SMT system has a generic pipeline that looks as described in Algorithm~\ref{algo:pbsmt}. 

\begin{algorithm}
\small
%\centering
\caption{Building a phrase-based system}
\label{algo:pbsmt}
\textbf{Input}: Parallel corpus between \emph{s} and \emph{t} \\
\textbf{Output}: A translation model ``tm'' 
\begin{algorithmic}[l]
	%\STATE{\textbf{Clean:}Pre-process both sides of the corpus} \label{aline:preprocess}
	\STATE{\textbf{Alignments: }Learn bi-directional alignments} \label{aline:alignments}
	\STATE{\textbf{Extraction: }Extract phrase pairs from alignments and compute likelihoods for each translation pair} \label{aline:scores}
	\STATE{\textbf{Tuning: }Set weights for features by maximizing BLEU score on a development set using MERT} \label{aline:MERT}
\end{algorithmic}
\end{algorithm}

Each step of Algorithm~\ref{algo:pbsmt} outlined above raises questions when faced with a low-resource language pair. Lets consider each step and discuss the problems that come up. Given parallel data, the goal of the alignment models~\cite{Brown:1993,Vogel:1996} is to learn which word in source language \emph{s} translates to target language \emph{t} and assign a likelihood to the pair. Facing corpus of small sizes, the models place higher likelihood on pairs seen fewer number of times. The phrase extraction step(Line~\ref{aline:scores}) looks at alignments learnt from Line~\ref{aline:alignments} in both directions and determines which phrases can translate from one language to another. At the end of this step, we have a phrase table which has rules of the following form : 

\begin{table*}
\small
\small
\centering
\begin{tabular}{p{0.3\linewidth}p{0.2\linewidth}p{0.4\linewidth}}
\toprule
src & tgt & features \\
\toprule
! la situacion de haiti & concerned about the situation in haiti & 0.5 8.16237e-09 1 0.000483004 2.718 \\
\bottomrule
\end{tabular}
\caption{Example of a phrase pair in the Haitian Kreyol to English table}
\label{table:example_rule}
\end{table*}

The table~\ref{table:example_rule}  says that the source phrase ``! la situacion de haiti , '' translate to the target phrase ``concerned about the situation in haiti ,'' with the feature values shown on the right. 

\begin{table*}
	\small
	\small
	%\centering
	\begin{tabular}{p{0.3\linewidth}p{0.6\linewidth}}
	\toprule
	Feature &  Explanation \\
	\toprule
	$p_{w}(f \mid e)$ & probability of seeing phrase ``f'' given ``e'' \\
	$p_{lex}(f \mid e)$ & lexical probability of seeing phrase ``f'' given ``e'' \\
	$p_{w}(e \mid f)$ & probability of seeing phrase ``e'' given ``f'' \\
	$p_{lex}(e \mid f)$ & lexical probability of seeing phrase ``e'' given ``f'' \\
   	phrase penalty & a constant value penalizing distortion \\
	\bottomrule
	\end{tabular}
	\caption{Features of the phrase pairs, where ``f'' is foreign/source \& ``e'' is target/english}
	\label{table:features}
\end{table*}

The 5 features are mentioned in Table~\ref{table:features}. The two $p_{w}$ are the phrasal features, features that determine the likelihood of the source phrase translating to target and vice-versa. The phrasal translation likelihood is computed by using relative frequencies, as shown in equation~\eqref{eq:trans}.

\begin{equation} \label{eq:trans}
	p_{w}(f \mid e) = \frac{c(f, e)}{\mathlarger{\sum\limits_{\grave{f}}}c(\grave{f}, e)}
\end{equation}

The counts referred to in equation~\eqref{eq:trans} are obtained from the alignments. Note that the alignment models that were learnt on the smaller corpus will cause some propagation of errors in the phrasal probabilities. 

 The lexical features~\cite{Koehn:03} are actually computed as shown in equation~\eqref{eq:lex} : 

\begin{equation} \label{eq:lex}
	p_{lex}(f \mid e, a) = \prod\limits_{i=1}^{n} \frac{1}{\{j | (i,j) \in a\}}
	\mathlarger{\sum\limits_{\forall (i,j) \in a}} w(f_{i} \mid e_{j})
\end{equation}


The intuition behind having a pair of lexical features is to reward syntactic phrases while penalizing spurious one. As shown in equation~\eqref{eq:lex}, the lexical probability is the product of the lexical alignment probabilities of the constituent words in the phrase table. Hence, if a longer source phrase aligns to an equally long target phrase, it can be penalized if the individual words are not aligned. 

Phrase-based SMT has been used with great success before in the literature. But, as described above, the approach is very data-driven and it is not clear how to negate the effectiveness of data and achieve fluent translations. 

The approach of triangulation~\cite{Cohn:07,Utiyama:07,Nakov:12} scales the feature values we compute using the smaller corpora using a translation model learnt from a much larger and cleaner corpus. A pattern in the previous studies is that a large translation model is used to augment an already large translation model. In a scenario like that, only the rich become richer.  

\section{Dissertation Outline}
\label{sec:outline}
Chapter~\ref{chapter:reality} describes the four low-resource languages we study followed by the models and results. \alert{add the chapter about Europarl}.

\section{Contributions}
\label{sec:summary}
We conduct the first in-depth study of triangulation, the first using four real low-resource languages with realistic data settings. As part of the dissertation, we also build the first translation systems for three of the four languages. Our best Haitian-Creole system outperform the best system from the Sixth Workshop on Machine Translation, 2011. 


\section{Setup}
\label{sec:setup}


Moses~\cite{Koehn:07} was used for all the experiments. To build our baseline systems, we followed the standard set of steps: generated bi-directional alignments using GIZA++ ~\cite{OchNey:03}, followed by phrase extraction using the --grow-diag-final-and heuristic. The decoder parameters were optimized using Minimum Error Rate Training ~\cite{Och:03} by minimizing BLEU loss on a development set. All scores reported are case-insensitive BLEU~\cite{Papineni:02}. All language models were generated using SRILM~\cite{Stolcke:02}.~\cite{Ken:11} was used for language model scoring when decoding.