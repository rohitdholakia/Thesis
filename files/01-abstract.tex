%!TEX root=/home/ska124/Dropbox/Thesis/thes-full.tex
%% Copyright 1998 Pepe Kubon
%%
%% `abstract.tex' --- abstract for thes-full.tex, thes-short-tex from
%%                    the `csthesis' bundle
%%
%% You are allowed to distribute this file together with all files
%% mentioned in READ.ME.
%%
%% You are not allowed to modify its contents.
%%

%%%%%%%%%%%%%%%%%%%%%%%%%%%%%%%%%%%%%%%%%%%%%%%%%
%
%       Abstract 
%
%%%%%%%%%%%%%%%%%%%%%%%%%%%%%%%%%%%%%%%%%%%%%%%%

\prefacesection{Abstract}

Triangulation refers to the use of a pivot language when translating from a source language to a target language. Previous research in triangulation has only focused on large corpora in the same domain.  This thesis conducts the first in-depth study on the use of triangulation for four real low-resource languages with realistic data settings, Mawukakan and Maninkakan, Haitian Kreyol and Malagasy, where fluent translations using statistical machine translation are difficult to obtain due to limited amounts of training data in the source-target language pair. We compare and contrast several design choices one needs to consider when using triangulation. We observe that triangulation via French improves translations significantly for Mawukakan and Maninkakan, two languages spoken in West Africa with little writing tradition. We also improve translations for real-world short messages sent in the aftermath of the Haiti earthquake in 2010 and news articles in Malagasy. This is the first in-depth study on triangulation for these four low-resource languages.

As part of the dissertation, we build the first effective translation system for three of these languages. We improve translation quality by injecting more data via pivot languages and show that in realistic data settings carefully considering triangulation design options is important. Furthermore, in all four languages since the low-resource language pair and pivot language pair data typically come from very different domains, we use insights from domain adaptation to fine-tune the weighted mixture of direct and pivot based phrase pairs to significantly improve translation quality. 





