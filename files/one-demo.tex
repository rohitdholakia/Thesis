%!TEX root=/home/ska124/Dropbox/Thesis/thes-full.tex
%% Copyright 1998 Pepe Kubon
%%
%% `one.tex' --- 1st chapter for thes-full.tex, thes-short-tex from
%%                the `csthesis' bundle
%%
%% You are allowed to distribute this file together with all files
%% mentioned in READ.ME.
%%
%% You are not allowed to modify its contents.
%%

%%%%%%%%%%%%%%%%%%%%%%%%%%%%%%%%%%%%%%%%%%%%%%%%%
%
%       Chapter 1 
%
%%%%%%%%%%%%%%%%%%%%%%%%%%%%%%%%%%%%%%%%%%%%%%%%

\chapter{Purpose}

This document outlines the functionality of the style file (package)
\textsf{csthesis.sty}\index{csthesis.sty@\textsf{csthesis.sty}},
suitable for writing theses in \LaTeXe{} at SFU.  The style file
modifies and generalizes the
\textsf{sfuthesis.sty}\index{sfuthesis.sty@\textsf{sfuthesis.sty}}, designed by
Margaret Sharon from ACS. The changes were done for two reasons: 1)~to
reflect the most recent regulations\index{thesis regulations} on SFU
theses;\footnote{You can get a printed copy on the 7th floor of the
  library or look at the online version at
  \texttt{http://www.lib.sfu.ca/kiosk/cwatts/REG-GYDS.htm}.}  and
2)~to improve the overall look of the theses. The changes will be
illustrated and commented on in Chapter~\ref{two}; the following
section summarizes the main features of the present package.


\section{Main features of \textsf{csthesis.sty}}

\begin{itemize}
\item fully conforms to the most recent regulations%
  \index{thesis regulations} on SFU theses;
\item offers a visually pleasing design;
\item supports all optional\index{optional material} thesis sections
  (Dedication, Quotation, Preface, etc.) and simplifies their
  inclusion;
\item supports the \verb+twoside+%
  \index{twoside option@\texttt{twoside} option} option of \LaTeX{}
  and coordinates it with two-sided printing\index{two-sided
    printing}---this is not allowed for the library copies, but you
  should definitely consider it for other copies;
\item comes with a full documentation\index{documentation}, describing
  all available commands and identifying which parts of the code can
  be modified and how to achieve specific effects (like
  adding/removing white space between signatures on Approval page);
\item comes with two files, illustrating the use of the package in
  writing theses (you are reading one of them). The following section
  describes all the files in the distribution in more detail.
\end{itemize}

\section{Distribution}

The style file
\textsf{csthesis.sty}\index{csthesis.sty@\textsf{csthesis.sty}} is
accompanied by several supporting files\index{auxiliary files}. These
are:

\begin{itemize}
\item \texttt{READ.ME}: the file containing the copying and
  distribution limitations, the installation instructions, the address
  where to sent bug reports, and the listing of all the files included
  in the \textsf{csthesis}%
  \index{csthesis bundle@\textsf{csthesis} bundle} bundle;
\item \texttt{csthesis.ins}\index{csthesis.ins@\texttt{csthesis.ins}}:
  the installation file for
  \textsf{csthesis.sty}\index{csthesis.sty@\textsf{csthesis.sty}}. The
  bundle comes unpacked and ready-to-use, so you don't need to
  generate \textsf{csthesis.sty} this way, but the file is included to
  allow for a more compact, packed version of the bundle.
\item \texttt{csthesis.dtx}\index{csthesis.dtx@\texttt{csthesis.dtx}}:
  the documented code for \textsf{csthesis.sty}. It lists all the user
  commands defined by the style file and explains the code. In
  particular, it points out the places where the code should be
  modified in case you run into troubles like not being able to fit
  the Approval section on one page.
\item
  \texttt{thes-short.tex}\index{thes-short.tex@\texttt{thes-short.tex}}:
  an example of a thesis written using \textsf{csthesis.sty}. The
  thesis is ``short'' in that it contains only the required parts of
  the front matter (Abstract, Acknowledgment, Contents) but no
  optional parts\index{optional material} (Dedication, Quotation, List
  of Figures, List of Tables, List of ``other things,'' Preface). In
  addition, the thesis does not contain the optional parts of the back
  matter---appendices and Index.
\item
  \texttt{thes-full.tex}\index{thes-full.tex@\texttt{thes-full.tex}}:
  an example of the ``full'' version of the same thesis, containing
  all the optional parts\index{optional material} from the front
  matter, several appendices, and Index.
\item the \texttt{files/}\index{files/@\texttt{files/}} subdirectory:
  the auxiliary files for \texttt{thes-short.tex} and
  \texttt{thes-full.tex}.
\item the \texttt{doc/}\index{doc/@\texttt{doc/}} subdirectory: the
  ``printouts'' of the documentation\index{documentation} for the
  package and for the two example theses.
\end{itemize}
 
You can use either of the two versions of the thesis as a guide in
designing your own thesis file---but it's in your interest to take a
look at both. The difference between the two versions is accomplished
by 1)~the switches located right under the beginning of the
\verb+document+ environment, 2)~inclusion/exclusion of the files
containing optional thesis parts\index{optional material} (Preface,
appendices, Index, etc.), and 3)~the inclusion of commands for
generating the List of ``other things'' in the preamble of
\texttt{thes-full.tex}\index{thes-full.tex@\texttt{thes-full.tex}}.

The two thesis files demonstrate the use of the commands provided by
the \textsf{csthesis.sty}\index{csthesis.sty@\textsf{csthesis.sty}}
package. If a more thorough explanation of some command(s) is
required, consult the
\texttt{csthesis.dvi}\index{csthesis.dvi@\texttt{csthesis.dvi}} guide.

\section{How to use \textsf{csthesis.sty}}

If you're starting writing from scratch\index{starting new thesis}, all you
need to do is to put the following two lines in the beginning of your
thesis document:
\begin{verbatim}
\documentclass[11pt]{report}%% or 10pt, or 12pt
\usepackage{csthesis}
\end{verbatim}
This makes all the commands defined in \textsf{csthesis.sty}%
\index{csthesis.sty@\textsf{csthesis.sty}} available to you; you can
consult the two examples theses and the
\texttt{csthesis.dvi}\index{csthesis.dvi@\texttt{csthesis.dvi}} to see
what commands\index{available commands} you have at your disposal and
how they should be used.

If you're upgrading%
\index{upgrading@upgrading from \textsf{sfuthesis.sty}} from
\textsf{sfuthesis.sty} to \textsf{csthesis.sty}, you also need to
include the two lines mentioned above. In addition, though, you need
to change some things in your thesis document because the two thesis
formats are not fully compatible%
\index{compatibility@compatibility with \textsf{sfuthesis.sty}}. In
particular, you should:
\begin{itemize}
\item replace the \verb+\afterpreface+ command of
  \textsf{sfuthesis.sty}\index{sfuthesis.sty@\textsf{sfuthesis.sty}}
  with \verb+\lists+ and \verb+\beforetext+, inserted in appropriate
  places (see the example theses);
\item in case you're having one or more appendices\index{appendices}, issue
  \verb+\appendix+ command after the last chapter of your thesis 
  before the first appendix. In addition, treat each appendix as a
  chapter, i.e., the main heading of the appendix should use the
  \verb+\chapter+ command.
\end{itemize}
These should be all changes you need to do---if there is more, let me
know (I haven't used \textsf{sfuthesis.sty}%
\index{sfuthesis.sty@\textsf{sfuthesis.sty}}, so I am to some extent
guessing here based on the coding differences between the two packages).
